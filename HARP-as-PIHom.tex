\section{HARP and partially injective homomorphisms}\label{sec:harp-as-pihom}

The connection between HARP and partially injective homomorphisms is a theoretical, yet useful one. If the HARP coarsenings could be restricted to partially injective homomorphisms or their compositions, general coarsenings could be learned for each task. In Section \ref{sec:harp-vs-pihom} we explore whether this restriction limits the performance of HARP.

When both the edge and star collapsing algorithms are used, the mapping \( \psi_i \) introduced in \ref{sec:graph-coarsening} is \textbf{not} a homomorphism nor a combination of homomorphisms due to its not meeting the injectivity condition on edges. However, it is met for the complementary graph (graph where edges are swapped with non-edges). For the complementary graph, the star collapsing algorithm is not a homomorphism, however, it can be replaced by homomorphisms that also collapses stars. We propose a coarsening that merges a hub node with half of its neighbours. Because merging a node with its neighbour only collapses an edge, this coarsening scheme is a composition of partially injective homomorphisms on the complementary graph. Such a model is studied in Section \ref{sec:harp-vs-pihom}.

This theoretical connection gives a way of constructing more general graph coarsenings by setting a constraint on such relations and finding the maximum of a subset of \( \mathcal{L} \) that satisfies such a constraint. The authors of \cite{schulz_mining_2019} present a way of effectively (in polynomial time for bounded tree-width graphs) searching for such maximally constrained homomorphisms.
