\section{HARP and the performance-complexity trade-off}\label{sec:performance-vs-complexity}

Graph-based methods such as node2vec typically have a large number of parameters - on the widely used OGBN-ArXiv dataset (see \cite{hu_open_2021}), the SOTA node2vec model has over 21 million parameters. At the same time, recent works have started to focus more heavily on simpler methods as a competitive alternative to heavy-weight ones (see \cite{frasca_sign_2020,huang_combining_2020,salha_keep_2019,zhang_eigen-gnn_2020}). As the authors of \cite{chen_harp_2018} observed, HARP improves the performance of models when fewer labelled data are available. The proposed lower complexity models based on HARP could also improve performance in a setting where only low fidelity data are available for large parts of the graph. Coarser models could be trained on them, with a subsequent training of finer models using only a limited sample of high fidelity data.

As a core principle of HARP, lower level representations of the task are generated and a model is learnt on them. How good these models are remains in question. In order to test this, several models were compared. Each model \( M_i \) was trained on graphs \( G_L, \dots, G_i \) as with HARP, then the embeddings were prolonged on graphs \( G_{i-1}, \dots, G_0 \) without training. On \( G_0 \), the models were then trained as they would be in an ordinary node2vec setup. With this schema, \( L \) models are obtained, each trained on graphs of different granularity. To examine the performance-complexity trade-off of HARP, the trade-off between decreasing predictive accuracy and decreasing amount of training data was evaluated.
