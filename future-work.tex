\section{Future work}

In this work-in-progress, two properties of graph-based models are studied. While these properties seem to be unconnected at first, they are both highly relevant to the future progress of our work. The target use of these approaches is the detection of similar sub-structures in graphs of very large size. It is computationally unfeasible to use techniques such as graph convolution for extremely large graphs. With HARP, such graphs could be significantly reduced in size, enabling the use of such more computationally demanding techniques. Also, the way in which the graphs are coarsened can be specifically tailored in a way that either preserves, or collapses the subgraphs of interest. Using such tailored coarsening techniques would allow for scanning for a pre-determined structure in the graph.

The target application of this work lies in the domain of computer network security. Malware running on endpoint devices connects to remote resources, i.e. as a check for internet connectivity or a connection to a Command \& Control server. From the traffic of regular user-induced behaviour as well as malware, a connection map can be built and represented with a graph. A skilled analyst can recognize nodes and edges associated with malicious software, however, in the present time, some families of malware are either sold to multiple threat actors and set up with duplicate infrastructure or they change the infrastructure they use as a means of evading detection. Our work aims towards recognizing such duplications or changes in infrastructure previously marked as malicious by an analyst. Moreover, the methods studied could enable automatic recognition of new infrastructure employed by previously seen malware families and suggest it for manual review, thus dramatically reducing the time needed until the new infrastructure is discovered. 
